\documentclass[letterpaper,12pt]{article}
\usepackage[spanish]{babel}
\spanishdecimal{.}
\selectlanguage{spanish}
\usepackage[spanish,onelanguage,ruled]{algorithm2e}
\usepackage[utf8]{inputenc}
\usepackage{graphicx}
\usepackage{caption}
\usepackage{subcaption}
\usepackage{hyperref}
\usepackage{verbatim}
\usepackage{amssymb}
\usepackage{mathtools}
\usepackage{amsmath}
\usepackage[natbibapa]{apacite}
\bibliographystyle{apacite}
%\usepackage[nottoc,numbib]{tocbibind}
\newcommand\ddfrac[2]{\frac{\displaystyle #1}{\displaystyle #2}}
\DeclareMathOperator{\atantwo}{atan2}

\title{Categoría AutoModelCar\\Torneo Mexicano de Robótica}
\author{Libro de Reglas, modalidad Presencial}
\date{Ciudad Victoria, 2022}
\begin{document}
\maketitle

%%%%%%%%%%%%%%%%%%%%%%%%%%%%%%%%%%%%%%%
%%%%%%%%% AGRADECIMIENTOS %%%%%%%%%%%%%
%%%%%%%%%%%%%%%%%%%%%%%%%%%%%%%%%%%%%%%
\section*{Agradecimientos}
Esta competencia inició gracias al proyecto “Visiones de Movilidad Urbana” con el cual se dotó de 32 vehículos a escala a diferentes universidades, institutos y centros de investigación del país. A nombre de todos los grupos de trabajo que recibieron un vehículo, en cualquiera de sus tres versiones, agradecemos al Dr. Raúl Rojas González, coordinador del proyecto, y a todos los académicos y autoridades que contribuyeron a su realización.  
 
\begin{flushright}
  \textit{
  El responsable técnico\\
  abril de 2022
  }
\end{flushright}

\section{Introducción}
El Torneo Mexicano de Robótica (TMR) es la competencia de robótica más importante de México que año con año reúne a estudiantes, profesores e investigadores. El objetivo principal es incentivar e impulsar la investigación y desarrollo de la robótica en México con miras a lograr un desarrollo integral de nivel internacional. Para lo anterior, el TMR incluye diferentes categorías de competición donde los equipos participantes ponen a prueba sus conocimientos y habilidades en la robótica.

Este año se abre nuevamente la categoría de vehículos a escala sin conductor donde se proponen varias tareas de conducción autónoma. En las dos primeras ediciones solo se permitieron los vehículos donados por la Universidad Libre de Berlín, con la finalidad de tener una plataforma estándar y tener así una evaluación más uniforme. Sin embargo, debido al interés que han mostrado diversos grupos de trabajo que no se vieron beneficiados con un vehículo a escala, este año se permitirán plataformas abiertas (vehículos a escala de construcción propia) bajo ciertas restricciones.

Los antecedentes directos del presente libro de reglas son el evento realizado en el mes de abril de 2017 en el IPN en la ciudad de México y las ediciones anteriores de la categoría AutoModelCar del Torneo Mexicano de Robótica. El presente documento pretende tomar las pruebas propuestas con adecuaciones mínimas y establecer el criterio de competencia que fortalezca la participación académica y estudiantil, así como el intercambio de experiencias en pro del desarrollo y capacitación de profesionales en esta área del conocimiento.

%%%%%%%%%%%%%%%%%%%%%%%%%%%%%%%%%%%%%%%
%%%%%%%% SOBRE LOS VEHÍCULOS %%%%%%%%%%
%%%%%%%%%%%%%%%%%%%%%%%%%%%%%%%%%%%%%%%
\section{Sobre los vehículos}


%%%%%%%%%%%%%%%%%%%%%%%%%%%%%%%%%%%%%%%
%%%%%%%%% SOBRE LA ARENA  %%%%%%%%%%%%%
%%%%%%%%%%%%%%%%%%%%%%%%%%%%%%%%%%%%%%%
\section{Sobre la arena}

%%%%%%%%%%%%%%%%%%%%%%%%%%%%%%%%%%%%%%%
%%%%%%%%%%%%% REGLAS %%%%%%%%%%%%%%%%%%
%%%%%%%%%%%%%%%%%%%%%%%%%%%%%%%%%%%%%%%
\section{Reglas}

%%%%%%%%%%%%%%%%%%%%%%%%%%%%%%%%%%%%%%%
%%%%%%%%%%%%% PRUEBAS %%%%%%%%%%%%%%%%%
%%%%%%%%%%%%%%%%%%%%%%%%%%%%%%%%%%%%%%%
\section{Pruebas}

%%%%%%%%%%%%%%%%%%%%%%%%%%%%%%%%%%%%%%%
%%%%%%%PUNTUACIÓN Y DESEMPATE %%%%%%%%%
%%%%%%%%%%%%%%%%%%%%%%%%%%%%%%%%%%%%%%%
\section{Sistema de puntuación y criterios de desempate}

%%%%%%%%%%%%%%%%%%%%%%%%%%%%%%%%%%%%%%%
%%%%%%%%%% CONTACTO %%%%%%%%%%%%%%%%%%%
%%%%%%%%%%%%%%%%%%%%%%%%%%%%%%%%%%%%%%%
\section{Contacto}

%%%%%%%%%%%%%%%%%%%%%%%%%%%%%%%%%%%%%%%
%%%%%%%%% CRÉDITOS %%%%%%%%%%%%%%%%%%%%
%%%%%%%%%%%%%%%%%%%%%%%%%%%%%%%%%%%%%%%
\section*{Créditos}
\end{document}

